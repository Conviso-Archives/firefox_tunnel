\documentclass[serif,mathserif]{beamer}
\usepackage{amsmath, amsfonts, epsfig, xspace}
\usepackage{algorithm,algorithmic}
\usepackage{pstricks,pst-node}
\usepackage{multimedia}
\usepackage[normal,tight,center]{subfigure}
\setlength{\subfigcapskip}{-.5em}
\usepackage{beamerthemesplit}
\usetheme{lankton-keynote}

\author[ ]{Using firefox to  make remote communication \quad \includegraphics[width=5.0cm]{img/firefoxtunnel.jpg}}

\title[firefox tunnel\hspace{2em}\insertframenumber/\inserttotalframenumber]{Firefox tunnel}

\date{ CoolerVoid - coolerlair@gmail.com - Dezember 21, 2017} %leave out for today's date to be insterted

\institute{Illustration by Anthony S Waters}

\begin{document}

\maketitle
% \section{Introduction}  % add these to see outline in slides
\begin{frame}
  \frametitle{whoami} 
CoolerVoid just another computer programmer and infosec guy.
\end{frame}

\begin{frame}
  \frametitle{Introduction}
  Motivations:\pause
  \begin{itemize}
    \item it's different technique, you can not find in msfvenom, veil...\pause
  \item RedTeam operations.\pause
  \item Improve the work.\pause
  \item Bypass any firewall. %leave out the \pause on the final item
  \end{itemize}
\end{frame}

% \section{Main Body} % add these to see outline in slides

\begin{frame}
  \frametitle{The Justify}
  \begin{itemize}
  \item \includegraphics[width=10.0cm]{img/tunnel1.png}
  \end{itemize}
\end{frame}

\begin{frame}
  \frametitle{The Justify}
  \begin{itemize}
  \item \includegraphics[width=10.0cm]{img/tunnel2.png}
  \end{itemize}
\end{frame}

\begin{frame}
  \frametitle{The Justify}
  \begin{itemize}
  \item \includegraphics[width=10.0cm]{img/tunnel4.png}
  \end{itemize}
\end{frame}

\begin{frame}
  \frametitle{The Justify}
  \begin{itemize}
  \item \includegraphics[width=10.0cm]{img/tunnel6.png}
  \end{itemize}
\end{frame}

\begin{frame}
  \frametitle{The Justify}
  \begin{itemize}
  \item \includegraphics[width=10.0cm]{img/tunnel8.png}
  \end{itemize}
\end{frame}

\begin{frame}
  \frametitle{How too -  Part 1}
  The recipe part 1 the web client:\pause
  \begin{itemize}
  \item Up in your remote host with httpd the directory firefox\char`_shell \pause
  \item Put that directory in root dir of server  example: /var/www/html \pause
  \item Set permissions to write and read in all files of dir... with chmod command...\pause
  \item if you enter in http://machine/firefox\char`_shell/firefox\char`_cmd\char`_tunnel.php?input=1 you can control remote server. %leave out the \pause on the final item
  \end{itemize}
\end{frame}

\begin{frame}
  \frametitle{How too -  Part 2}
  The recipe part 2 - the firefox tunnel server:\pause
  \begin{itemize}
  \item You need mingw32, gcc, c++ and make to build.\pause
  \item At file firefox\char`_tunnel.cpp change line 20 in var domain, replace it to your remote machine IP or DNS. \pause
  \item compile in root dir of server with command "mingw32-make", take exe file in directory bin \pause
  \item Execute the file, control him  using the PHP client. \pause
  \item if you enter in http://machine/firefox\char`_shell/firefox\char`_cmd\char`_tunnel.php?input=1 you can control remote  server.  
  \end{itemize}
\end{frame}

% \section{Conclusion} % add these to see outline in slides
\begin{frame}
  \frametitle{Demo}
  \begin{itemize}
  \item At YouTube
  \item Look that following:
  \item https://www.youtube.com/watch?v=C23N4yDRkjU
  \end{itemize}
\end{frame}

\begin{frame}
  \frametitle{Credits}
  \begin{itemize}
  \item Thank you
  \item Any doubt ? talk to me coolerlair@gmail.com
  \item Twitter: @Cooler\char`_freenode
  \item Github: https://github.com/CoolerVoid/
        %http://newsgroups.derkeiler.com/Archive/Comp/comp.text.tex/2007-11/msg00299.html
  \end{itemize}
\end{frame}

\end{document}
